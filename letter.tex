% !TeX program = lualatex
\documentclass[10pt]{article}
\pagestyle{empty}

\usepackage{fontspec}
\usepackage{unicode-math}
\setmainfont{EB Garamond}
\setmathfont{STIX Two Math}

\usepackage[letterpaper,margin=1in]{geometry}
\usepackage{parskip}
\setlength{\parskip}{1.0em}
\usepackage{enumitem}

\begin{document}
\raggedright

% \begin{flushright}
%     Zaheen Farraz Ahmad\\
%     ID: 1402435\\
%     \date{}
% \end{flushright}
\hfill
\begin{minipage}{0.3\textwidth}
    \raggedright
    {\large\textsc{Zaheen Farraz Ahmad}}\\
    Student ID: 1402435\\

    \vspace{1em}

    \today
\end{minipage}
\vspace{4em}

\textsc{Recipient Full Name}\\
Address
\vspace{1em}

\textbf{\textsc{Subject:}} This is where the subject line should go.
\vspace{1em}

Recipient,

The first paragraph should contain a brief introduction and a quick overview of
the subject matter at hand. Use this paragraph to explain why the letter is
being written and why you are contacting the recipient --- for example, an
inquiry, a complaint, or a request.

The body should be the paragraph(s) containing the details pertinent to the
subject of the letter. The body should be concise and exact, so the information
should be the bare essentials. Concentrate on clarity and a well-organized,
logical flow.

The final paragraph should state the action you would like to see from the
request i.e., what you expect to gain from the recipient.


\vspace{2em}

Sign-off,

\vspace{6em}

Zaheen Farraz Ahmad

\end{document}
